%=============================================================================80
%	                    Basix LaTex Template  			       %
%==============================================================================%
\documentclass{article}
%==============================================================================%
%	                          Packages                                     %
%==============================================================================%
% Packages
\usepackage[utf8]{inputenc}
\usepackage{graphicx}
\usepackage{amsmath}
\usepackage{amssymb}
\usepackage{braket}
\usepackage{float}
\usepackage{subcaption}
\usepackage[margin=0.7in]{geometry}
\usepackage[version=4]{mhchem}
\usepackage{lipsum}
%==============================================================================%
%                           User-Defined Commands                              %
%==============================================================================%
% User-Defined Commands
\newcommand{\be}{\begin{equation}}
\newcommand{\ee}{\end{equation}}
\newcommand{\benum}{\begin{enumerate}}
\newcommand{\eenum}{\end{enumerate}}
\newcommand{\pd}{\partial}
\newcommand{\dg}{\dagger}
%==============================================================================%
%                             Title Information                                %
%==============================================================================%
\title{Title}
\date{4/3/18}
\author{Name}
%==============================================================================%
\begin{document}

\maketitle

\section{Stuff}
\lipsum{1-2}

\section*{No Numbering}
\lipsum{1-2}

\subsection{A subsection}
Simple Equations

\begin{equation}
\hat{A} f(x,y,\hdots) = B(x,y,\hdots)
\end{equation}

Or using the User-Defined Command Above

\be
\hat{A} f(x,y,\hdots) = B(x,y,\hdots)
\ee

Labeling an equation, for example equation \ref{eq:homo_ex} is a homogenous equation. 
\be \label{eq:homo_ex}
\frac{d^2y}{dx^2} + d\frac{dy}{dx} + y = 0
\ee

We can also split up and align equations
\be
\begin{split}
    \sum_{n=0}^N a_nx^{(n)} &= 0\\
    \sum_{n=0}^N a_n \left(e^{\lambda t}\right) ^{(n)} &= 0\\
    \sum_{n=0}^N a_n \lambda^n e^{\lambda t} &= 0\\
    e^{\lambda t} \sum_{n=0}^N a_n \lambda^n &= 0\\
    \sum_{n=0}^N a_n \lambda^n &= 0\\
\end{split}
\ee

And the all important figures. 
\begin{figure}[h]
  \centering
  \includegraphics[scale=0.7]{my_fig.png}
    \caption{My Figure Caption}
  \label{fig:under_damped}
\end{figure}

\begin{figure}[H]
    \centering
    \begin{subfigure}[b]{0.49\textwidth}
        \includegraphics[width=\textwidth]{subfig.png}
  	\caption{The Morse Potential.} 
    \end{subfigure}
    \begin{subfigure}[b]{0.49\textwidth}
        \includegraphics[width=\textwidth]{another.png}
        \caption{The Piecewise Potential.}
    \end{subfigure}
    \caption{Potentials for Linear ODEs with Variable Coefficients}
\end{figure}

\end{document}
