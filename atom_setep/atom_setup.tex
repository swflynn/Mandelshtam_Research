\documentclass[preprint,showpacs,preprintnumbers,amsmath,amssymb]{revtex4}
\usepackage{graphicx}% Include figure files
\usepackage{dcolumn}% Align table columns on decimal point
\usepackage{bm}% bold math
\newcommand{\be}{\begin{equation}}
\newcommand{\ee}{\end{equation}}
\newcommand{\pd}{\partial}
\def\eps{\varepsilon}
\def\D{\displaystyle}               % display style
\def\half{{1\over 2}}               % display style
\def\att{                   % mark at the margin
    \marginpar[ \hspace*{\fill} \raisebox{-0.2em}{\rule{2mm}{1.2em}} ]
    {\raisebox{-0.2em}{\rule{2mm}{1.2em}} }
        }



\begin{document}

\section*{Atom}
Notes for setting up the Atom ide. (9-19-19)
Once you install atom you can open it from the terminal with \$ atom.
Go to edit, preferences on the top bar of the gui to find various options.

Please Note; with these changes if you run into packages not working, try
completly exiting atom and then restart (it may fix your issues).
If you previously installed atom and stopped using it, it may be easier to
uninstall and reinstall atom itself (this helped me).

\section{Packages}
One convenient thing about atom is the number of packages available for easy
download.
Go to edit, preferences, packages and simply type in the package name in the
search bar.
Once you install the package you can click on its settings to get more
information, or to change the default settings.

You can look online for various packages, or simply search through the Packages
gui to see what is available.
Install and uninstall packages at will through the gui it will warn you if
something is not properly working on your system.

\section{KeyBindings}
A very useful part of atom is how easy it is to customize.

A useful thing to do is change your shortcut keybindings for commands you find
yourself using a lot.
If you go to the top of the gui and click the Packages tab you will find a list
of your packages.
Hovering the selections gives the default hotkeys for each package.
If you wish to change these or any of the default KeyBindings do the following.

Go to settings, keybindings (this shows all the keybindings for your system).
At the top of the file you will find a link to your keymap file which will open
a file called keymap.scon
In this file you can over-ride the default keybindings to whatever you want.
On the keybindings page the folder symbol on the far left of each command
(keystroke column) contains a copy link.
If you click this it copies the command syntax, which you can paste into your
keymap.cson file.
Once in the keymap.cson file just change the keybinding to whatever you like.
If there is an issue with the keybinding you choose it will tell you when you
try to save the file.

Note: On my system when I change the keymap.cson file I need to completly close
and re-open atom for it to take affect.
If your chagnes are not working try that.

\section{Themes and Syntax}
Start with colors, go to Setting Themes to select your color schemes.
UI is the actual interface and syntax is for files.
You should choose a color scheme that is convenient for your eyes (remember in
a dark room choose dark colors, light room light colors).

There are some default themes and syntax installed on atom, but you can find
plenty online.

Once you are happy go to the install tab, and to the right of the search bar
select themes and type whatever theme you want.

For UI I like the default atom light (this comes intalled you don't need to go
find it).
For syntax I like pen-paper-coffee (light colors).

Once you install your UI and Syntax return to the Themes tab and simply select
your new choices.

\section{Using VIM in Atom}
I have used VIM for the past 10 years so I need to use its commands when working
on a project.
Feel free to skip this section if you do not know VIM, I don't suggest new
programmers bother learning it.
VIM is very powerful, but it takes a long time to become efficient (you have
been warned).

\subsection{vim-mode-plus}
Start by downloading this package, it gives the basic vim commands.

\subsection{vim-mode-plus-keymaps-for-surround}
Sets default keybindings.

\subsection{relative-numbers}
Sets the active line to absolute line number, and all other lines to relative
distance.
Useful for navigating through the file by number.

\subsection{lazy-motion}
A nice searching command.

\section{Python}
This is a simple list for python that is not exhaustive by any means.
I use the terminal to run my python codes, however, look into the script package
if you want to compile within atom itself.

autocomplete is a defualt package in atom which provides autocomplete support
for various languages.
If you are working with python you can get their specific ide.

\subsection{ide-python}
Standard Python ide.

\subsection{autocomplete-python}
package for autocomplete, feel free to use kite (machine learning to
autocomplete) or the local choice.

\subsection{python-debugger}
Debugger in atom for python.
alt-shft-r sets breakpoints in your code.
alt-r will bring up a debugger menu with a gui where you can step through the
code and debug.

\section{Fortran}
If you want to work with Fortran you need to download a few packages first.

Note: you need python and the fortls (fortran language server).
If you run into problems make sure you have fortls (install with pip), and
give the absolute path to fortls in the ide-fortran package settings.

\subsection{atom-ide-ui}
First we need to install the atom-ide-ui package.
This package gives language support to various ides in atom.

\subsection{language-fortran}
Next we need the language-fortran package which gives nice syntax for fortran.

\subsection{ide-fortran}
Finally install the ide-fortran package.

\section{Latex}
If you want to work with Latex, it is possible to build in atom, and there are a
few options.
I use the terminal to do these commands, so this is a quick solutions I found.

\subsection{Atom-LaTex}
This package seems to be a stable build environment for atom-latex.
By default it assumes you have every tex file (like a .bib file), so if
you are not using  all files it will throw an error when you compile.

To fix this go to Atom-LaTex package settings.
In the toolchain setting choose custom toolchain.
Then in the custom toolchain commands do something like TEX ARG DOC
(just google this problem to find custom toolchains).
%\%TEX \%ARG \%DOC && \%TEX \%ARG \%DOC

\section{Useful Packages}
The whole reason to use atom is that there are tons of packages that exist and
can be easily installed from atom to optimize your work flow.
Here is a growing list of useful packages for me.

\subsection{platformio-ide-terminal}
Allows you to control multiple terminal windows from atom.
The terminal is located in the bottom left corner and can be given a name by
right clicking.

\subsection{atom-file-icons}
nice package that adds images to the atom project tab showing what type of file
it is.

\subsection{advanced-open-file}
Allows your to make new files/directories quickly using the keyboard.

\subsection{Atom Beautify}
Takes your code and applies proper typesetting, it works for a ton of file types
.

\end{document}
